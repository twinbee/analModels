%% Assignment #1 for markov chains

\documentclass[10pt,fullpage]{article}

\usepackage{amsmath,amssymb,amsthm,amsfonts} % Typical maths resource packages

\usepackage{graphics}                 % Packages to allow inclusion of graphics
\usepackage{graphicx}

\usepackage{hyperref}                 % For creating hyperlinks in cross references

\topmargin -1.5cm
\oddsidemargin -0.04cm
\evensidemargin -0.04cm
\textwidth 16.00cm
\textheight 23.50cm
\parskip 7.2pt
\parindent 0.0in

\makeindex

\title{ Analytical Models Assignment I }

\author{Matthew Bennett \\
{\small\em \copyright \  Draft date \today }}

 \date{ }

\begin{document}
\maketitle

(1) Solve the system of linear equations:\\\\
(a) $\pi_{1} = .7\pi_{1} + .3\pi_{2} + .2\pi_{3}$\\\\
(b) $\pi_{2} = .2\pi_{1} + .5\pi_{2} + .6\pi_{3}$\\\\
(c) $\pi_{3} = .1\pi_{1} + .2\pi_{2} + .2\pi_{3}$\\\\
(d) $1 = \pi_{1} + \pi_{2} + \pi_{3}$\\\\

Since the above equations are all linearly independent, (c) will not be used to solve the unknowns. Using the Gauss-Jordan method, our initial matrix is as follows:\\

$\left(%
\begin{array}{cccc}
  -\frac{3}{10} & \frac{3}{10} & \frac{2}{10} & 0 \\
  \frac{2}{10} & -\frac{5}{10} & \frac{6}{10} & 0 \\
  1 & 1 & 1 & 1 \\
 \end{array}%
\right)$\\

Swap row 1 and row 3:\\
$\left(%
\begin{array}{cccc}
  1 & 1 & 1 & 1 \\
  \frac{2}{10} & -\frac{5}{10} & \frac{6}{10} & 0 \\
  -\frac{3}{10} & \frac{3}{10} & \frac{2}{10} & 0 \\
 \end{array}%
\right)$\\\\



$ Row2 \leftarrow -\frac{2}{10}Row1 + Row2$\\
$ Row3 \leftarrow \frac{3}{10}Row1 + Row3$\\
$\left(%
\begin{array}{cccc}
  1 & 1 & 1 & 1 \\
  0 & -\frac{7}{10} & \frac{4}{10} & -\frac{2}{10} \\
  0 & \frac{6}{10} & \frac{5}{10} & \frac{3}{10} \\
 \end{array}%
\right)$\\



$ Row2 \leftarrow -\frac{10}{7}Row2$\\
$\left(%
\begin{array}{cccc}
  1 & 1 & 1 & 1 \\
  0 & 1 & -\frac{4}{7} & \frac{2}{7} \\
  0 & \frac{6}{10} & \frac{5}{10} & \frac{3}{10} \\
 \end{array}%
\right)$\\



$ Row3 \leftarrow -\frac{6}{10}Row2 + Row3$\\
$\left(%
\begin{array}{cccc}
  1 & 1 & 1 & 1 \\
  0 & 1 & -\frac{4}{7} & \frac{2}{7} \\
  0 & 0 & \frac{59}{70} & \frac{9}{70} \\
 \end{array}%
\right)$\\

\newpage

$ Row3 \leftarrow \frac{70}{59}Row3$\\
$\left(%
\begin{array}{cccc}
  1 & 1 & 1 & 1 \\
  0 & 1 & -\frac{4}{7} & \frac{2}{7} \\
  0 & 0 & 1 & \frac{9}{59} \\
 \end{array}%
\right)$\\


$ Row1 \leftarrow -Row3 + Row1$\\
$ Row2 \leftarrow \frac{7}{4}Row3 + Row2$\\
$\left(%
\begin{array}{cccc}
  1 & 1 & 0 & -\frac{9}{59} \\
  0 & 1 & 0 & \frac{22}{59} \\
  0 & 0 & 1 & \frac{9}{59} \\
 \end{array}%
\right)$\\


$ Row1 \leftarrow -Row2 + Row1$\\
$ Row2 \leftarrow \frac{7}{4}Row3 + Row2$\\
$\left(%
\begin{array}{cccc}
  1 & 0 & 0 & \frac{28}{59} \\
  0 & 1 & 0 & \frac{22}{59} \\
  0 & 0 & 1 & \frac{9}{59} \\
 \end{array}%
\right)$    Done. \\

\begin{equation}
\pi_{1} = \frac{28}{59}, \pi_{2} = \frac{22}{59}, \pi_{3} = \frac{9}{59}
\end{equation}

\newpage

(2) Explain how these values are derived:\\

\textbf{"the steady state probability of a sunny day is .47,"}\\
From the array obtained in the previous problem, we know that the steady state probability for a sunny day is 28/59, which is approximately 0.47457627118644067796610169491525. Marsan rounded down.\\

\textbf{"the mean recurrence time of sunny days is 2.1 days,"}\\
Mean recurrence time for state j is given by the following formula (2.26) for an ergodic system \begin{equation}\pi_{j}=\frac{1}{M_{j}}\end{equation} Therefore, $ M_{j} = \frac{1}{\pi_{j}} = \frac{59}{28} \approx 2.1$.\\

\textbf{"the average number of sunny days between two consecutive rainy days is 3.11,"}\\
This should be the same as (2.34 and Bayes' Rule) $\frac{\pi{1}}{\pi{3}} \approx 3.11$

\textbf{"if today the sun is shining, we can expect 2.33 more rainy days before the weather changes,"}\\
The average number of steps spent in state i before going to another state is described by a geometric process where the expected wait time to leave state i (2.36) is \begin{equation}E[W_{i}] = \frac{1}{1 - p_{ii}}\end{equation}. One Sunny day has already passed, so the number of sunny days remaining until a non-Sunny say is $E[W_{sunny}] = \frac{1}{1 - 0.7} - 1 \approx 2.33 $


(3) Give a new piece of information.

The fourth-step transition probability matrix is

$P^{4} = P^{2}P^{2} = \left(%
\begin{array}{ccc}
  .4917 & .3598 & .1485 \\
  .4612 & .3831 & .1557 \\
  .4540 & .3886 & .1574 \\
 \end{array}%
\right)$

Therefore, the probability that the fourth day after a sunny day is 0.4917, and other facts can be garnished from that matrix.

\end{document}
