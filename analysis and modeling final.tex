%% FINAL PROJECT FOR ANALYTICAL MODELS
%% last modifed 05/08/06
\documentclass[times,12pt,fullpage]{article}

\usepackage{hyperref}
\usepackage{times}
\usepackage{amsmath,amssymb,amsfonts} % Typical maths resource packages

\topmargin 0.0in \oddsidemargin -0.2in \evensidemargin 0.8in
\textwidth 6.80in \textheight 8.6in
\parskip 7.6pt
\parindent 0.25in

\makeindex
\usepackage{fancyhdr}
\thispagestyle{fancy} \pagestyle{fancy} \fancyhf{}
\lhead{Multiprocessor Modeling Guide} \chead{} \rhead{Bennett
\thepage} \lfoot{} \cfoot{} \rfoot{}

\title{Modeling Guide for Multiprocessing Systems}
\author{Matthew Bennett \\
{\small School of Computing, University of Southern Mississippi} \\
{\small {\em matthew.bennett@usm.edu} \ Typeset in \LaTeX on \
\today{} } }
\date{ }

\begin{document}
\maketitle
\begin{abstract}\noindent A tremendous variety of techniques exist for the modeling and
subsequent simulation of multiprocessor systems. This paper presents
several of the approaches of Marsan et al. \cite{marsan}, in a
condensed format suitable to practical systems engineer or computer
scientist. Most of these techniques are based upon Generalized
Stochastic Petri Nets. The guide covers Bucci's Preemptive Time Nets
\cite{bucci} to fill in gaps where continuous-time models fail.
\end{abstract}

\section*{Preliminaries}
A working knowledge of multiprocessor architecture, statistical
distributions, and the modeling technique of General Stochastic
Petri Nets is required. For the latter, Murata \cite{murata}
provides a succinct and quick introduction. Markov chains are
mentioned, but the results presented do not require thorough
understanding of the device, since the work has already been
completed and proven by others.

Many of the graphs or numerical results in Marsan \cite{marsan}, and
\cite{murata} are similar enough to be asymptotically bound for some
constant. The purpose of this guide is to provide a straightforward
and efficient means of guiding the systems architect through the
modeling process and on into either simulation or analysis. The
reader may check those sources cited for more in-depth coverage of
very detailed models, since only the simplest models are presented
within.

A few taxonomic niceties will save the designer some time. The
modeling phase has been split into five categories (as in
\cite{marsan}). After the model is developed, the developer may move
on to modeling fault tolerance (\ref{fault tolerance}) or
verification (\ref{analysis}).

\section{Bus-Contention Free Architectures}
The most trivial multiprocessor system to model is one that
experiences no contention whatsoever. The canonical example in
academia (as in \cite{marsan}) is a crossbar-connected switch
between $n$ processor elements and $m$ memory elements. An analog of
crossbar switch is the Plain Old Telephone System (POTS) exchange.
These are rarely seen in practice, since the number of switches and
interconnections increases by a factor of $\Theta(mn)$, making the
interconnection network extremely expensive for even small numbers
of processors and memories.

Marsan \cite{marsan} mentions that most time spent in any
multiprocessing time is idle time due to contention for a common
resource such as a bus. It is therefore no surprise that modeling
contention-free systems is strait-forward in terms of waiting. The
analysis instead concentrates on statistical modeling of random
activity within the system. This is easily accomplished using
queueing networks, where processors are nodes without queues, and
memory modules are nodes with finite queues \cite{marsan}. For a
crossbar architecture, there is an edge from every processor element
to every memory element, since any processor may read any memory
element regardless of other processors. Some simplification occurs
when two or more processors try to access the same memory node in a
given time step: only one is served. Simulation can be performed
using the Monte Carlo method, with memory accesses being produced
using a random variable (``Markovian process'' is exponential) on
all producers (processors) and consumers (memories) in the net.

As more becomes known about the system under analysis, more can be
included in the modeling queueing network. Marsan \cite{marsan}
provides a useful ontology by asking if the devices are synchronous,
or asynchronous, and also whether memory accesses are uniform or
non-uniform between processors.

Marsan gives a number of case studies. The simplest one is
attributed to Bhandarkar \cite{bhan}, who assumes that memory
accesses are uniform across processors, with lost requests in the
case of memory contention. Bandarkar creates a vector of states that
the queueing network can take on, and then further simplifies the
amount of information at hand by creating super-states, or {\em
equivalence classes} of mutually indistinguishable states (since
processors and memories are assumed identical). He derives a formula
(Marsan fig 6.2) which can be used to algorithmically calculate
super-state transition probabilities in a Markov process, and Marsan
(p. 124, 136) gives further approximation formulas from others' work
which can be calculated in better time on a computer. These
approximations are not as important as they once were, because
computers are substantially more capable today than they were in
1986, at the time of publication. They are all valid, as they are
equivalent to the exact result obtained by Bandarkar \cite{bhan}. A
good approximation closed-form for the number of busy memory modules
$ \beta $ is given by Rau \cite{rau} , using binomial approximations
for all events (equation \ref{raus}). In this formula, $m$ counts
memories, and $p$ counts processors.
\begin{equation}\label{raus}
\frac{\sum_{i=0}^{\text{min}(m,p)-1} 2^i \binom {m-1}{i} \binom
{p-1}{i}}
{\sum_{i=0}^{\text{min}(m,p)-1}\frac{2^i}{i+1}\binom{m-1}{i}\binom{p-1}{i}}
\end{equation}
Delay models, in which memory contention requests are queued instead
of dropped, were also considered my multiple authors. Many of the
results were similar, with memory utilization expectedly being
better with the addition of queues to memory \cite{marsan} (136).
Rau's formula given above is an excellent minimum bound for back of
the envelope calculations for networks without bus contention.

\section{Shared Memory Systems}

Most real systems are not free of bus contention, because of the
nonlinear overhead cost of building complete connection networks.
When bus contention is introduced, the location of the bus with
respect to the memory elements and processors plays a vital role in
the performance and contention of the system as a whole. In this
section, we investigate the subset of systems for which a number of
processors each have a private local memory, connected by a local
bus, and must contend for access to a shared memory, which requires
control of one of one or more global buses. This multiprocessor
architecture is common to many multiprocessor, multi-core
workstations as well as supercomputers such as the classic Cray
series. Classical multi-threading problems, including deadlock
prevention and avoidance, starvation, and mutual exclusion must all
be taken into account for any shared memory systems. Marsan
\cite{marsan} lays out a number of simplifying assumptions for
modeling shared memory systems. Namely, they are: no delay when
capturing the bus (other than normal bus contention), CSMA/CA bus
discipline, and immediate release of the bus upon completion (as
with capture). Marsan uses the term ``active state'' to mean that a
processor is not currently accessing memory, ``waiting state'' to
mean that a processor is waiting for the bus or memory, and ``access
state'' to mean that the processor is currently in memory.

\subsection{Single-Bus Shared Memory Systems}
The architecture is defined by Marsan to be a single global bus
connecting a number of local buses to the shared memory resource.
Each local bus serves a single processor and its private memory.

He chooses three values to take into consideration: access times,
active times, and

\subsubsection{Exponential Distribution}
The Exponential Distribution represents . In a generalized
stochastic petri net, a single exponentially distributed random
event can be represented by a place connected to a timed transition,
where the firing rate of that transition takes on the parameter
$\lambda$ of the exponential distribution, and determines the
likelihood of that even firing depending on a continuous time
variable \cite{marsan}. An example of an Exponentially distributed
random variable is likelihood of a webserver receiving a packet at a
particular time.

\subsubsection{Erlang-k Distribution}
An Erlang-K distribution models k Exponentially distributed random
events which depend upon one another. They must occur in {\em
serial}. In the Petri Net representation, Erlang-K events should be
represented by a k-chain of exponential General Stochastic
Continuous-Timed Petri events \cite{marsan}. An example of an Erlang
distributed random variable is the number of packets to a web server
at a particular instant of time.

\subsubsection{Hyper-Exponential Distribution}
A hyper-exponentially distributed random variable counts the number
of simultaneous occurrences of an exponentially distributed random
event, such as processors trying to access the same piece of memory
simultaneously.

\subsubsection{Queueing disciplines}

Marsan \cite{marsan} defines three queueing disciplines for dealing
with memory access contention, and mutual exclusion assurance. Fixed
priority means that the processors have some preordained priority
for which can preempt one another for memory access. Process sharing
is like round-robin in that processors take their turn accessing
memory in the order they attempt to get in, but they are guaranteed
access because of some global device, like in time sharing. First
come first serve simply operated like a FIFO queue, but provides no
guarantee that a process will give up the bus to allow others in.

All of the combinations of these ideas can be analyzed, and have
been in several different papers. Again, see Marsan p. 157 for a
full bibliography of works. A simplified queuing network model or
Markov chain was successfully employed to deal with simple cases in
which access times and/or active times were equally exponentially
distributed, but Petri nets were used by Marsan to get results for
the more complex models where distribution was hyper-exponential,
Erlang, or more complicated.

In all cases, the numerical results of Marsan seem to indicate that
modeling with a generalized stochastic Petri net can produce
approximate simulation results fairly close to the analytical
results for simpler single-bus shared memory systems. The numerical
results show that the queueing discipline is not an important
consideration when modeling small systems. Since Petri nets are more
graphical and more intuitive than analytical methods, and because
the mathematics become complicated for very complex systems, the
recommended method for modeling anything complex is Generalized
Stochastic Petri nets as described in the 2nd half of Marsan
\cite{marsan} (Chapter 7 and the beginning of Chapter 8).

\subsection{Multi-Bus Shared Memory Systems}
Multi-bus Shared Memory systems can be modeled with Petri nets using
the same techniques as single-bus shared memory systems. Generally,
the number of buses are represented as tokens in a resource pool,
which is a place. Depending upon the architecture of the system, and
because of Marsan's simplifying assumptions for shared memory
systems, immediate transitions usually follow the global bus
resource pool (capturing and releasing being instantaneous, but
active, access times being exponential).

Marsan gives a full analytical treatment to many complicated
multi-bus systems, but that is unnecessary since most systems can be
easily intuitively modeled with Petri nets. Marsan et al. first
obtains upper and lower bounds limiting the gain or loss of
processing power, so any numerical results given by Petri net
modeling should fall within the domain on \cite{marsan} (p.
186-189). This is generally a good methodology, since most
bottlenecks in this configuration are assumed to stem from global
bus contention.

\section{Distributed Memory Systems}
Shared memory systems are not the only type of distributed system. A
true distributed system uses a paradigm more like message passing.
This architecture consists of one or more global buses connecting
local buses. Each local bus contains a processor and at least one
memory element. Unlike previous shared memory systems, there is no
memory connected to the global bus(es). Instead, the memory at each
local bus is used by any processor in the system, and both local and
global buses must be secured at each access step. An example of this
type of system is a computer network utilizing file sharing. Because
of the high dynamism of such networks, analysis of any but the
simplest distributed memory systems is impossible with queueing
nets. Marsan \cite{marsan} proposes tens of architectures that are
both single global bus and multiple global bus and also incorporate
distributed memory, but does not give any indication of extending
this methodology to generalized distributed systems.

To model any distributed memory systems, a Petri net should be
employed, and numerical results obtained. The reachability tree of
the Petri net can then be used to discover some analytical artifacts
of the system, such as whether it is live, as described in
\cite{murata}.

\section{Extended Petri Nets}

Bucci \cite{bucci} describes an extended version of Petri nets where
the continuous time model is replaced with discrete time events. A
global clock of some sort . A ``firing event'' occurs whenever the
random variable distribution coincides {\em as well as} the time has
come for that transition to fire. The idea of this extension is that
a Petri net can operate within a discrete time domain, following a
global clock, while keeping the ability to induce transitions from
state-to-state concurrently (ie using threads or a similar
mechanism). This dramatically increases the simulation uses of GSPN
models, but does not really add much to their descriptive power.
Bucci's method should be taken into account any time that a
simulation must occur, especially in a simulation with lots of
simultaneity.

\section{Fault Tolerance Modeling}\label{fault tolerance}
If systems include repairable components, a GSPN event (place /
timed transition) can represent the break and rapair of any
component in the system, just as

\section{Analysis and Verification}\label{analysis}
Generally, any sort of Petri Net model is good for modeling a
specific system, but can only provide numerical results. In order to
assure that those results are statistically significant, many runs
(at least 30) should be performed in simulation, and every variable
must be checked for consistency with a real system. One technique
which is liberally applied by many \cite{murata} \cite{marsan} is to
use a verifiable Markovian process or queueing network to provide
upper and lower analytical bounds. At this point, the system can
simply be built and run. Analysis can be extremely difficult, so
most of the time the Petri net approach will be more useful for the
budding system developer, without rigorous mathematical background.

\begin{thebibliography}{5}

\bibitem{marsan} Marsan, N. Balbo, G. Conte, G. "Stochastic Petri Nets." Performance Models of Multiprocessor Systems. Cambridge: MIT Press. 1986. pp. 72 - 98.

\bibitem{bucci}  Bucci, G. Correctness Verification and Performance Analysis Using Stochastic Preemptive Time Petri Nets. IEEE Transactions on Software Engineering. Vol 23, No 11. November 2005. pp. 913 - 937.

\bibitem{murata} Murata, T. Petri Nets: Properties, Analysis, and Applications. Invited Paper. Proceedings of the IEEE. Vol 77. No 4. April 1989. pp. 541 - 579.

\bibitem{bhan} Bhandarkar, D. Analysis of Memory Interference in
Multiprocessors. IEEE Transactions on Computers Vol 24. No 9.
September 1975. 897-908.

\bibitem{rau} Rau, B. Interleaved Memory Bandwidth in a Model of a Multiprocessor Computer System. IEEE Transactions on Computers Vol 28. No 9.
September 1979. 678 - 681.


\end{thebibliography}

\end{document}
